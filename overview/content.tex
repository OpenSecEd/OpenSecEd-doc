\mode*
\section{Introduction}

We can strongly argue that security and privacy becomes more important as our 
society become more and more dependent on new technologies.
As such, the security and privacy training of engineers becomes more important 
than ever.

In 2015, I started to open source my teaching material and started working on 
the Open Security and Privacy Education\footnote{%
  URL:\@ \url{https://github.com/OpenSecEd}.
} project (the OpenSecEd project).
The aim of this project is to provide better teaching material which is open
to anyone and each topic covered in the material should be strongly connected 
to the research --- both in the field of each topic, but the material design 
should also be founded research (pedagogy and didactics).
The openness is important for three reasons: first, that anyone can take part 
of it; second, that anyone can help keep it up-to-date with the progress of the 
research on each topic; third, others can use the material and reproduce the 
scientific evaluation of the teaching design.

Unlike many textbooks on security, the material should have privacy at
its core, \eg privacy-by-design, and also have a strong emphasis on usability 
--- instead of treating these subjects separately.
This will help shape the future engineers to improve today's situation; where 
privacy (and security) is only added as an afterthought, where we use passwords 
instead of solutions like anonymous credentials.
Similarly, the material should give a wider view of the security field.
Many of the building blocks that have been available for decades, \eg anonymous
credentials, remain unused.
The students must see that there is more than just encryption and digital 
signatures, so they think outside current practices.

With similar reasoning, it's also important that the scientific methods of the 
security field are visible to students.
Partly so that they can adapt a scientific mindset, to question and evaluate 
things properly.
But also so that they get the habit of searching for already researched 
solutions themselves.

While the contents of the material itself is founded in the research literature 
of the security and privacy fields, the design of the material should be 
founded in the research literature of the fields of pedagogy and didactics.
It's the intention to have a research-based approach in designing the material 
and thus this project can contribute to the research in the field of didactics, 
\eg by reporting how the design of the material affects students' learning.


\subsection{What is the problem?}

\dots

\begin{frame}
  \begin{itemize}
    \item \dots
  \end{itemize}
\end{frame}


%%% REFERENCES %%%

\begin{frame}[allowframebreaks]
  \printbibliography{}
\end{frame}
